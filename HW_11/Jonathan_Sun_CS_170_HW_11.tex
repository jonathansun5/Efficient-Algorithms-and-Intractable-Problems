% Search for all the places that say "PUT SOMETHING HERE".

\documentclass[11pt]{article}
\usepackage{amsmath,textcomp,amssymb,graphicx,enumerate,hyperref,enumitem,mathtools,tikz-qtree,listings,tikz,txfonts}
\definecolor{light-gray}{gray}{0.85}
\lstset{
    numbers=left,
    breaklines=true,
    backgroundcolor=\color{light-gray},
    tabsize=2,
    basicstyle=\ttfamily,
    literate={\ \ }{{\ }}1
}

\def\Name{Jonathan Sun}  % Your name
\def\SID{25020651}  % Your student ID number
\def\Homework{11} % Number of Homework
\def\Session{Fall 2017}


\title{CS170 --- \Session --- Homework \Homework \space Solutions}
\author{\Name, SID \SID}
\markboth{CS170 --- \Session --- Homework \Homework \space --- \Name}{CS170 --- \Session --- Homework \Homework --- \Name}
\pagestyle{myheadings}
\date{}

\def\endproofmark{$\Box$}
\newenvironment{proof}{\par{\bf Proof:}}{\endproofmark\smallskip}
\newenvironment{reduction}{\par{\bf Reduction:}}{\smallskip}
\newenvironment{FourPartSolution}{\par{\bf Four-Part Solution:}}{\smallskip}
\newenvironment{mainIdea}{\par{\bf Main Idea:}}{\smallskip}
\newenvironment{pseudocode}{\par{\bf Pseudocode:}}{\smallskip}
\newenvironment{proofOfCorrectness}{\par{\bf Proof of Correctness:}}{\endproofmark\smallskip}
\newenvironment{runTime}{\par{\bf Run Time:}}{\smallskip}
\newenvironment{justification}{\par{\bf Justification:}}{\smallskip}
% \newenvironment{proofOfCorrectness}{\par{\bf Proof of Correctness:}}{\endproofmark\smallskip}
% \newenvironment{runTime}{\par{\bf Run Time:}}{\smallskip}
% \newenvironment{justification}{\par{\bf Justification:}}{\smallskip}

\usepackage[margin=1in]{geometry}



\begin{document}
\maketitle

\section*{0. Who Did You Work With?}

Collaborators: Kevin Vo, Aleem Zaki, Jeremy Ou



\newpage
\section*{1. True or False}
For this question, I will do the first problem, which is:
\vspace*{1\baselineskip}
\\
Given a positive integer $n$, if for all $a \in \{1 ... n - 1\}$ such that $(a, n) = 1$ we have that $a^{n - 1} \equiv 1 \text{ mod } n$ then this implies that $n$ is a prime.
\vspace*{1\baselineskip}
\\
The statement above is false and I will justify this by giving a counter example. A counter example is to use a Carmichael number for $n$. This Carmichael number $n$ is a composite number and $a$ will be a number that is relatively prime to $n$. An example of this will be $n = 561$ and $a = 2$. $n = 561$ is a composite number because $561 = 3 \times 11 \times 17$ and $a = 2$ is relatively prime to $561$ because it does not share any factors with $561$ except for $1$. Therefore, I will get:
\begin{align*}
2^{561 - 1} \equiv 1 \text{ mod } 561 \\
2^{560} \equiv 1 \text{ mod } 561
\end{align*}
Noting that $561 = 3 \times 11 \times 17$, the equation above is true because of the following:
\begin{itemize}
\item
For $3$ as the mod, $2^{2} \equiv 1 \text{ mod } 3$. So, $2^{560} = \left(2^{2}\right)^{280} \equiv 1^{280} \equiv 1 \text{ mod } 3$.
\item
For $11$ as the mod, $2^{10} \equiv 1 \text{ mod } 11$. So, $2^{560} = \left(2^{10}\right)^{56} \equiv 1^{56} \equiv 1 \text{ mod } 11$.
\item
For $17$ as the mod, $2^{16} \equiv 1 \text{ mod } 17$. So, $2^{560} = \left(2^{16}\right)^{35} \equiv 1^{35} \equiv 1 \text{ mod } 17$.
\end{itemize}
Since $2^{560} \equiv 1 \text{ mod } 3 \equiv 1 \text{ mod } 11 \equiv 1 \text{ mod } 17 $, $2^{560} \equiv 1 \text{ mod } 561$. Therefore, I have shown that $a^{n - 1} \equiv 1 \text{ mod } n$ for $a = 2$ and $n = 561$. However, the statement also says that $n$ should also be a prime. This is not the case because $n = 561 = 3 \times 11 \times 17$. Therefore, I have shown that the statement is False.



\newpage
\section*{2. Proof}
Given two primes $p$ and $q$ and an $a$ such that $(a, pq) = 1$, I can show that $a^{(p - 1)(q - 1)} \equiv 1 \text{ mod } pq$ by letting $n = pq$ so that $a^{(p - 1)(q - 1)} \equiv 1 \text{ mod } n$ such that $(a, n) = 1$.
\begin{itemize}
\item
For $p$ as the mod, $a^{(p - 1)(q - 1)} = \left(a^{p - 1}\right)^{q - 1}$. From Fermat's Little Theorem, we know that $a^{p - 1} \equiv 1 \text{ mod } p$ so $\left(a^{p - 1}\right)^{q - 1} \equiv 1^{q - 1} \equiv 1 \text{ mod } p$. Therefore, $a^{(p - 1)(q - 1)} \equiv 1 \text{ mod } p$.
\item
For $q$ as the mod, $a^{(p - 1)(q - 1)} = \left(a^{q - 1}\right)^{p - 1}$. From Fermat's Little Theorem, we know that $a^{q - 1} \equiv 1 \text{ mod } q$ so $\left(a^{q - 1}\right)^{p - 1} \equiv 1^{p - 1} \equiv 1 \text{ mod } q$. Therefore, $a^{(p - 1)(q - 1)} \equiv 1 \text{ mod } q$.
\end{itemize}
So, we get that $a^{(p - 1)(q - 1)} \equiv 1 \text{ mod } p$ and $a^{(p - 1)(q - 1)} \equiv 1 \text{ mod } q$. Therefore, by the Chinese Remainder Theorem, there must exist an unique solution for $a^{(p - 1)(q - 1)} \text{ mod } pq$ and since $a^{(p - 1)(q - 1)} \equiv 1 \text{ mod } pq$ is a solution, this must also be unique. Therefore, I have shown that $a^{(p - 1)(q - 1)} \equiv 1 \text{ mod } pq$.
\end{document}
