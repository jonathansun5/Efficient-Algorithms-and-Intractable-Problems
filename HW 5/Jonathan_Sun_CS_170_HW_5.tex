% Search for all the places that say "PUT SOMETHING HERE".

\documentclass[11pt]{article}
\usepackage{amsmath,textcomp,amssymb,graphicx,enumerate,hyperref,enumitem,mathtools,tikz-qtree,listings,tikz}
\definecolor{light-gray}{gray}{0.85}
\lstset{
    numbers=left,
    breaklines=true,
    backgroundcolor=\color{light-gray},
    tabsize=2,
    basicstyle=\ttfamily,
    literate={\ \ }{{\ }}1
}

\def\Name{Jonathan Sun}  % Your name
\def\SID{25020651}  % Your student ID number
\def\Homework{5} % Number of Homework
\def\Session{Fall 2017}


\title{CS170 --- \Session --- Homework \Homework \space Solutions}
\author{\Name, SID \SID}
\markboth{CS170 --- \Session --- Homework \Homework \space --- \Name}{CS170 --- \Session --- Homework \Homework --- \Name}
\pagestyle{myheadings}
\date{}

\def\endproofmark{$\Box$}
\newenvironment{proof}{\par{\bf Proof:}}{\endproofmark\smallskip}
\newenvironment{FourPartSolution}{\par{\bf Four-Part Solution:}}{\smallskip}
\newenvironment{mainIdea}{{\bf Main Idea:}}{\smallskip}
\newenvironment{pseudocode}{\par{\bf Pseudocode:}}{\smallskip}
\newenvironment{proofOfCorrectness}{\par{\bf Proof of Correctness:}}{\endproofmark\smallskip}
\newenvironment{runTime}{{\bf Run Time:}}{\smallskip}
\newenvironment{justification}{\par{\bf Justification:}}{\smallskip}
% \newenvironment{proofOfCorrectness}{\par{\bf Proof of Correctness:}}{\endproofmark\smallskip}
% \newenvironment{runTime}{\par{\bf Run Time:}}{\smallskip}
% \newenvironment{justification}{\par{\bf Justification:}}{\smallskip}

\usepackage[margin=1in]{geometry}



\begin{document}
\maketitle

\section*{0. Who Did You Work With?}

Collaborators: Kevin Vo, Aleem Zaki, Jeremy Ou



\newpage
\section*{1. Scheduling Homeworks}
\begin{enumerate}[label=(\alph*)]
\item


\item


\item


\end{enumerate}


\newpage
\section*{2.  Graph Coloring}
\begin{enumerate}[label=(\alph*)]
\item
Following the hint, allow $p(G):G\mapsto\mathbb{N}_0$ to denote the potential of a graph $G$. Let this potential be explicitly defined as the number of edges in $E$ that have the same color vertex on either end (same-color edges), which by definition always takes on a value $\in \mathbb{N}_0$. Each time that the loop runs, a vertex $v\in B$ is changed from one color to the other.  $\forall v \in B$ there are $\geq 85$ vertices adjacent to $v$ and of the same color, and there are at most $85$ vertices adjacent to $v$ of the opposite color. This shows that even with the maximal number of non-similar and minimal similar adjacent vertices there is still a decrease in the number of same-color edges  by at least one (namely, one of the edges with $v$ as an end) when you switch the color of $v$. Thus, each iteration of the loop leads to a decrease of the value of $p(G)$ by at least one.

\item


\end{enumerate}



\newpage
\section*{3. 170-Graph}
\begin{mainIdea}
\\
ok
\end{mainIdea}
\\
\begin{runTime}
\\
ok
\end{runTime}



\newpage
\section*{4. Weighted Set Cover}





\newpage
\section*{5. Quaternary Huffman}
\begin{mainIdea}
\\
ok
\end{mainIdea}
\\
\begin{proofOfCorrectness}
\\
ok
\end{proofOfCorrectness}



\newpage
\section*{6. Simple and Naive Cluster Analysis}


















\end{document}
