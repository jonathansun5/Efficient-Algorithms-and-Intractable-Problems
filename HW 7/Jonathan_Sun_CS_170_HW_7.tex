% Search for all the places that say "PUT SOMETHING HERE".

\documentclass[11pt]{article}
\usepackage{amsmath,textcomp,amssymb,graphicx,enumerate,hyperref,enumitem,mathtools,tikz-qtree,listings,tikz}
\definecolor{light-gray}{gray}{0.85}
\lstset{
    numbers=left,
    breaklines=true,
    backgroundcolor=\color{light-gray},
    tabsize=2,
    basicstyle=\ttfamily,
    literate={\ \ }{{\ }}1
}

\def\Name{Jonathan Sun}  % Your name
\def\SID{25020651}  % Your student ID number
\def\Homework{7} % Number of Homework
\def\Session{Fall 2017}


\title{CS170 --- \Session --- Homework \Homework \space Solutions}
\author{\Name, SID \SID}
\markboth{CS170 --- \Session --- Homework \Homework \space --- \Name}{CS170 --- \Session --- Homework \Homework --- \Name}
\pagestyle{myheadings}
\date{}

\def\endproofmark{$\Box$}
\newenvironment{proof}{\par{\bf Proof:}}{\endproofmark\smallskip}
\newenvironment{FourPartSolution}{\par{\bf Four-Part Solution:}}{\smallskip}
\newenvironment{mainIdea}{\par{\bf Main Idea:}}{\smallskip}
\newenvironment{pseudocode}{\par{\bf Pseudocode:}}{\smallskip}
\newenvironment{proofOfCorrectness}{\par{\bf Proof of Correctness:}}{\endproofmark\smallskip}
\newenvironment{runTime}{\par{\bf Run Time:}}{\smallskip}
\newenvironment{justification}{\par{\bf Justification:}}{\smallskip}
% \newenvironment{proofOfCorrectness}{\par{\bf Proof of Correctness:}}{\endproofmark\smallskip}
% \newenvironment{runTime}{\par{\bf Run Time:}}{\smallskip}
% \newenvironment{justification}{\par{\bf Justification:}}{\smallskip}

\usepackage[margin=1in]{geometry}



\begin{document}
\maketitle

\section*{0. Who Did You Work With?}

Collaborators: Kevin Vo, Aleem Zaki, Jeremy Ou



\newpage
\section*{1. A HeLPful Introduction}
\begin{enumerate}[label=(\alph*)]
\item
The necessary and sufficient conditions on real numbers $a$ and $b$ that make the linear program infeasible do not exist. This is because no matter what $a$ and $b$ are, $x$ and $y$ can both equal $0$ and this will always satisfy the constraints.



\item
The necessary and sufficient conditions on real numbers $a$ and $b$ that make the linear program unbounded are if $a < 0$ or $b < 0$. This is because the conditions to maximize $x + y$ and $ax + by \leq 1$ with either $a < 0$ or $b < 0$ would make the region's area infinite. In other words, if the slope of the line for $ax + by \leq 1$ is positive, then there is no maximum for either $x$ or $y$ making the region's area infinite. If the slope of the line is negative, then both $x$ and $y$ will have individual maximum values and so $x + y$ will be bounded. Therefore, the to make the linear program unbounded, either $a$ or $b$ will need to be less than zero to make the slope positive.



\item
The necessary and sufficient conditions on real numbers $a$ and $b$ that allow the linear program to have a unique optimal solution are that $a > 0$, $b > 0$, and $a \neq b$. This allows for the slope of the line for $ax + by \leq 1$ to be negative, which will give individual maximum bounds to both $x$ and $y$. Furthermore, $a \leq b$ makes the slope not equal to $-1$, which would cause the maximum value of $x$ to equal the maximum value of $y$. This would give multiple solutions since $max x + y$ would all be the same value if the slope is $-1$. Therefore, we need the slope to be negative and also not equal to $-1$.
\end{enumerate}



\newpage
\section*{2. TeaOne}
\begin{enumerate}[label=(\alph*)]
\item
The cost of creating a ZestyJuice is $5 \times 0.1 + 1 \times 0.2 + 8 \times 0.01 = 0.78$. Since a ZestyJuice sells for $4.5$, the profit of selling one is $4.5 - 0.78 = 3.72$. The cost of creating a MilkTea is $12 \times 0.1 + 16 \times 0.2 = 4.4$. Since a MilkTea sells for $5$, the profit of selling one is $5 - 4.4 = 0.6$. With this, I will represent ZestyJuice as $z$ and MilkTea as $m$. The linear program will be:
\begin{align*}
\text{max } 3.72 z + 0.6 m
\end{align*}
\begin{align*}
z \leq 60
\end{align*}
\begin{align*}
m \leq 40
\end{align*}
\begin{align*}
0.78 z + 4.4 m \leq b
\end{align*}
\begin{align*}
z \geq 0 \text{, } m \geq 0
\end{align*}



\item
\begin{align*}
\text{min } 60 y_1 + 40 y_2 + 6 y_3
\end{align*}
\begin{align*}
y_1 + 0.78 y_3 \geq 3.72
\end{align*}
\begin{align*}
y_2 + 4.4 y_3 \geq 0.6
\end{align*}
\begin{align*}
y_1 \geq 0 \text{, } y_2 \geq 0 \text{, } y_3 \geq 0
\end{align*}
\item
There are three types of solutions for this part since we can have a budget that is so large that we can afford to make all $60$ ZestyJuice and $40$ MilkyTea, a limited budget so we maximize ZestyJuices since they make the most profit, and another limited budget where we maximize MilkyTea because they turn a profit.
\begin{align*}
z = \frac{b} {0.78} \text{, } m = 0 \text{ for } b < 46.8
\end{align*}
\begin{align*}
z = 60 \text{, } m = \frac{b - 46.8} {4.4} \text{ for } 46.8 \leq b \leq 222.8
\end{align*}
\begin{align*}
z = 60 \text{, } m = 40 \text{ for } b > 222.8
\end{align*}
\end{enumerate}



\newpage
\section*{3. Mountain pass}
\begin{enumerate}[label=(\alph*)]
\item




\item






\end{enumerate}



\newpage
\section*{4. The Hungry Caterpillar}




\newpage
\section*{5. Star-shaped polygons in 2D}
\vspace*{1\baselineskip}
The idea for this problem is that the point $x$ will be able to see all the points of the polygon, which means that there can be a direct line between each point of the polygon to the point $x$ and none of these points will intersect or be the same equation as each other between each point of the polygon and $x$. In other words, the linear program needs to show whether there exists a point $x$ that is the intersection of all the lines that go from each point of the polygon to $x$. This can further "simplified" by having the linear program show whether there exists a point $x$ that is the intersection of all the lines that go from each CORNER of the polygon to $x$. \\
\vspace*{1\baselineskip}
Therefore, the linear program can be represented as:
\begin{align*}
\text{max } None
\end{align*}
\begin{align*}
a_i x + b_i y > c_i \text{ where } i \in \{1, 2, ..., n - 1\}
\end{align*}



\newpage
\section*{6. All Knight-er}
\begin{FourPartSolution}
\\
\begin{mainIdea}
\\
ok
\end{mainIdea}
\\
\begin{pseudocode}
\begin{lstlisting}
ok
\end{lstlisting}
\end{pseudocode}
\begin{proofOfCorrectness}
\\
ok
\end{proofOfCorrectness}
\\
\begin{runTime}
\\
ok
\end{runTime}
\\
\begin{justification}
\\
ok
\end{justification}
\end{FourPartSolution}
\end{document}
